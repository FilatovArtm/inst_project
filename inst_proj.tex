\documentclass[a4paper, 12pt]{article}
\usepackage{fontspec} 
\usepackage[utf8]{inputenc}
\usepackage[russian,english]{babel}
\setmainfont{Times New Roman}


 % для компиляции теперь нужно использовать XeLaTex 
\usepackage{amsfonts}
\usepackage[paper=a4paper,top=13.5mm, bottom=13.5mm,left=20mm,right=20mm,includefoot]{geometry}
\usepackage{indentfirst}
\usepackage{dsfont}
\usepackage{graphicx} 
\graphicspath{{images/}} 
\setlength\fboxsep{3pt} 
\setlength\fboxrule{1pt} 
\usepackage{wrapfig} 
\usepackage{amsmath}
\usepackage{amsthm}
\renewcommand{\proofname}{Доказательство}

\DeclareMathOperator*{\argmin}{arg\,min}
\newcommand{\inde}{\perp\!\!\!\perp}


\theoremstyle{definition}
\newtheorem{definition}{Определение}[section]


\theoremstyle{plain}
\newtheorem{lemma}{Лемма}[section]
\newtheorem{svoistvo}{Свойство}[section]
\newtheorem{theorem}{Теорема}[section]


\title{Исследовательский проект по инстуциональной экономике}
\date{\today}

\begin{document}
\maketitle

\noindent{\bf Аннотация }-- ...

\section{Введение}
Мы устверждаем, что в существуют рынки на которых присутствует сразу несколько игроков со стороны спроса на труд в одном регионе. 
\section{Обзор литературы}

Обозреть статью с моделью по монопсонии. 
Обозреть стаью по олигпсонии, там сделать акцент на том, что олигпосния может многие аномалии объяснить. (Возможно нужно будет также сделать акцент на взаимодействии )
Обозреть эконометрическую статью


\section{Модель}

Многие статьи рассматривают академический рынок труда  как монопсонию (ссылки на статьи). Мы утверждаем, что существуют также рынки олигопсонического типа и совсем не очевидно, что на них будет наблюдаться тенденция к понижению зароботной платы.  Мы рассмотрим модель олигпсонического рынка труда в рамках которой у профессора будет выбор либо остаться в своем регионе и пойти в один из университетов, либо уехать, но понести издержки переезда. Модель будет строится на основе модели монопсонического рынка из статьи Seniority and Monopsony in the Academic Labor Market
Michael R. Ransom (1993). Из неё мы возьмём ...  

Пусть существует рынок в рамках одного города, где присутствует $n$ университетов, предъявляющие спрос на одних и тех же академических работников. Обычно в США в одном городе находится лишь несколько ВУЗов, которые могли бы предъявлять спрос и на одну и ту же рабочую силу, но для общности мы пока буем предполагать, что их $n$. Рынок академических работников основан в своем большинстве на личном взаимодействии, так как преподаватели являются единственными экземплярами предлагающие именно такую комбинацию способностей. Это сильно отличает академический рынок труда от многих других (неплохо бы ссылку). В следствии этого, каждый университет выбирает какую зарплату предложить отдельно для каждого работника. Зарплату предложенную $i$ университетом мы будет обозначать $w_i$. Также будет существовать рыночная заралата $w_m$, именно с такой зарплатой университету придется нанять работника, если профессор, которому предложили зарплату $w_i$ откажется. Университет минимизирует свои ожидаемые издержки: 
\[
EC(w_i) = p_i(w_m, w_i, m, w_{j \neq i})w_i + (1 - p_i(w_m, w_i, m, w_{j \neq i}))w_m
\]

Как мы видим, в зависимости от предложенной зарплаты, зарплаты предложенной другими университетами, зарплаты по рынку и издержкам переезда будет устанавливаться некоторое распределение, случайным исходом для которой будет являться выбор, который сделает преподаватель. Как и в модели с монопсонией мы будем предлагать для простоты, что агенты обладают полной информацией, и единственный источник случайности заключается в выборе профессора одного из университетов. Мы можем предложить, что функция $p_i ( w_m, w_i, m, w_{j \neq i}) $ имеет следующий вид: 
\[
p_i( w_m, w_i, m, w_{j \neq i}) = \frac{p_i(w_i - w_m + \delta m) w_i}{\#\{w_j : w_i = w_j\}}\mathds{1} \{ w_i \geq w_ {i \neq j}\}
\]
Профессор сравнивает предложенную ему зарплату некоторым университетом с зарплатой рынка и своими дисконтированными издержками. Более того, в виду наличия функции индикатора максимальной цены, преподаватель будет рассматривать лишь тот университет, который предложил ему наибольшую зарплату. Всем остальным вузов придется нанимать преподавателя на общем рынке. В виду присутствия ненаблюдаемых факторов решение всё равно остается случайным. 

Возможен также случай, когда несколько университетов предложат максимальную зарплату. Именно этот фактор учитывает знаменатель дроби, который означает, что вероятность попасть в конкретный ВУЗ делится на количество ВУЗов предложивших такую же зарплату. 

Проанализируем поведение стороны спроса на рынке. В статье Olygopsony and Monopsonistic Competition in Labor Markets Bhaskar, A. Manning, Ted To (2002)  рассматриваются различные варианты взаимодействия игроков на олигпсоническом рынке.  Мы будем предполагать, что они не вступают в сговор. Мы считаем, что такая предпосылка вполне оправдана, так как профессора являются единичным товаром, в следствии чего доход от его приобретения почти невозможно разделить легальным путем. 

Заметим, что если вуз устанавливает зарплату ниже чем его конкуренты, то он сталкивается с рыночной ценой с вероятностью 1, соответсвенно в симметричном равновесии  вузы должны установить одинаковую зарплату. И должно быть выполнено следующее условие для каждого конкретного учебного заведения
\[
p_i(w_i - w_m + \delta m) w_i + (1 - p_i(w_i - w_m + \delta m))w_m = \frac{p_i(w_i - w_m + \delta m) w_i}{n} + (1 -\frac{ p_i(w_i - w_m + \delta m)}{n})w_m \leq w_m
\]
Условие означает, что ВУЗу не выгодно увеличивать зарплату, чтобы повысить вероятность прихода к нему профессора. %Возможно это не верные рассжудения.
Раскрывая скобки мы получаем 
\[
\frac{(n-1)p_i(w_i - w_m + \delta m) w_i}{n} = \frac{(n-1) p_i(w_i - w_m + \delta m) w_m}{n}
\]
\[
w_i = w_m
\]

Мы видим, что в случае такой конкуренции будет установлено равновесие в котором каждый вуз предлагает одинаковую зарплату равную рыночной, ученый в свою очередь с вероятностью $p(\delta m)/n$ выбирает один из вузов, и с вероятностью $1 - p(\delta m)$ уезжает из города. Соответсвенно мы не ожидаем увидеть на данных тенденцию к понижению зарплаты на таких рынках. Более того, можно сказать, что почти не существует ситуаций, когда несколько ВУЗов в одном городе предлагают профессору рыночную зарплату. Но так как теперь ВУЗам безразлично с точки зрения модели, какого профессора брать, то со стороны спроса может остаться только один агент. Естественно предположить, что не за каждого профессора идёт такая борьба на рынке труда, следовательно на данных мы ожидаем увидеть не противоположный эффект, как в предыдущих статьях, но более слабый. 


\section{Кейс}

\section{Заключение}




\end{document}



