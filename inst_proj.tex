\documentclass[a4paper, 12pt]{article}
\usepackage{fontspec} 
\usepackage[utf8]{inputenc}
\usepackage[russian,english]{babel}
\usepackage{natbib}
\setmainfont{Times New Roman}
\renewcommand{\baselinestretch}{1.5} 

\usepackage[unicode,colorlinks=false,urlcolor=blue,hyperindex,breaklinks]{hyperref} 

 
\usepackage{amsfonts}
\usepackage[paper=a4paper,top=13.5mm, bottom=13.5mm,left=20mm,right=20mm,includefoot]{geometry}
\usepackage{indentfirst}
\usepackage{dsfont}
\usepackage{graphicx} 
\graphicspath{{images/}} 
\setlength\fboxsep{3pt} 
\setlength\fboxrule{1pt} 
\usepackage{wrapfig} 
\usepackage{amsmath}
\usepackage{amsthm}
\renewcommand{\proofname}{Доказательство}

\DeclareMathOperator*{\argmin}{arg\,min}
\newcommand{\inde}{\perp\!\!\!\perp}

\parindent=1.2cm
\theoremstyle{definition}
\newtheorem{definition}{Определение}[section]


\theoremstyle{plain}
\newtheorem{lemma}{Лемма}[section]
\newtheorem{svoistvo}{Свойство}[section]
\newtheorem{theorem}{Теорема}[section]


\title{Исследовательский проект по инстуциональной экономике}

\begin{document}
\begin{center}
{\Large\sc Исследовательский проект по инстуциональной экономике}\vspace{0.2cm}\\
\end{center}
\begin{center}
{\bf Филатов Артём, Федотова Мирослава, Киселёв Дмитрий и Фурсов Иван}\vspace{0.1cm}\\
{\it Национальный исследовательский университет Высшая Школа Экономики\\ 
Москва, Российская Федерация}\vspace{0.1cm}\\
\end{center}


\noindent{\bf Аннотация }— Основная идея работы заключалась в построении модели олигпсонического рынка труда на основе случайных исходов для проверки гипотезы об отрицательной зависимости стажа работы в одном университете и заработной платы профессоров в США. После построения модели был получен вывод: на рынках, где присутствует несколько университетов одного уровня, мы ожидаем увидеть более слабый эффект от увеличения стажа работы в конкретном ВУЗе, нежели в случае монопсонии. Это в первую очередь связано с конкуренцией за профессоров, которая переодически будет возникать на рынках подобного типа. Выводы модели были проверены на данных о зарплатах профессоров в городе Хьюстон, Техас. 

\section{Введение}
Во второй половине 20 века было замечено, что увеличение стажа профессора на одном рабочем месте ведёт к снижению заработной платы.  Это довольно необчный феномен  для рынка труда, поскольку, как правило, наблюдается абсолютно противоположный эффект --- стаж положительно влияет на зарплату. В результате был поставлен вопрос о причинах данного явление, ответ на которой с нашей точки зрение абсолютно не тривиален и открыт. Более того можно в принципе поставить под сомнение наличие данного эффекта, как и было сделано в статье \cite{3}. Большинство статей написанных по данной тематике предполагает наличие монопсонической структуры рынка. Мы в свою очередь считаем, что некоторые локальные рынки можно рассматривать как олигпсонию, например если в городе находится несколько университетов, то они могут предъявлять спрос на одну и ту же рабочую силу. Целью нашей работы было понять будет ли в случае олигопсонии наблюдаться такой же эффект. В случае его наличия мы хотели сравнить силу обратной зависимости, которую ученые наблюдают на монопсонических рынках, и результаты нашей модели. И в завершении проверить полученные результаты на данных университетов, которые можно рассматривать как олигопсонию. В основу работы легла модель, предложенная в \cite{1}, которую мы модифицировали, включив в неё конкурирующие между собой университеты. Случайный подход использовавшийся в данной модели позволяет учитывать различные ненаблюдаемые факторы влияющие на выбор университета профессором. Мы пытались проанализировать равновесие, которое будет образовываться на таких рынках и динамику зарплат профессора. Для проверки модели на реальных данных мы использовали информацию о системе Хьюстонских университетов, которые принадлежать примерно одному уровню, в следствие чего могут рассматриваться как олигопсония. 

\section{Обзор литературы}

Большинство теорий, связанных с рынком труда, предполагает наличие положительной зависимости уровня заработных плат и продолжительности времени работы в конкретном месте. Наиболее известными работами, демонстрирующими данный эффект, являются статьи Walter Oi (1962) и Jacob Mincer (1974), описывающие теорию специфического человеческого капитала. Согласно ей фирма поощряет работника за приобретение навыков и знаний, которые специфичны для конкретного работодателя. Фирма устанавливает более высокую заработную плату, стимулируя работника не увольняться, чтобы «окупить» собственные инвестиции в человеческий капитал. Всё же существуют такие рынки труда, где исследователи выявляют отрицательную зависимость уровня заработных плат и продолжительности времени работы в конкретном месте. Именно таковым и является академический рынок труда, привлёкший наше внимание.

Сперва обратимся к модели, описанной в статье Michael R. Ransom (1993)\footnote{Michael R. Ransom. Seniority and monopsony in the academic labor market seniority and monopsony in the academic labor market. The American Economic Review, 83(1):221–233, 1993.}. Автор привёл эмпирическое доказательство того, что с увеличением времени работы профессоров в одном университете их заработная плата падает. Он продемонстрировал выводы трёх различных исследований, получивших несильно разнящиеся результаты, подтверждающие его основную гипотезу. Первое из них - Carnegie Commission National Survey of Higher Education: Faculty Study (1969) заключило что, индивиды, проработавшие как минимум 30 лет в одном университете, зарабатывают примерно на 15\%  меньше тех, кто проработал менее двух лет. При этом эффект ярче выражен для более престижных университетов. Важно отметить, что регрессия строилась с поправкой на учёную степень, научную область, пол, расу, качество института, диплом которого получен, продолжительность контракта и контролируемость нанимающего учреждения частным сектором. 

	Результаты исследования American Council on Education's (1973) были следующими:  для исследовательских университетов заработная плата снижается примерно на 0.4\% с ежегодным ростом сеньорити (условимся так называть продолжительность времени работы в конкретном месте). 
	
	Другое исследование, упомянутое автором, - Survey of the American Professoriate (1977). Для всей выборки университетов корреляция между заработными платами и сеньорити не являлась существенной, однако, для исследовательских университетов заработные платы действительно снижались в ростом сеньорити. Эффект наиболее выражен для профессоров с 16-20 годами сеньорити, которые зарабатывают примерно на 7\% меньше нежели те, у кого сеньорити не более двух лет (с поправкой на опыт, образование и другие факторы). 
	
	По результатам всех трёх исследований заключается, что смена места работы может увеличить заработную плату профессора примерно на 5-6\%. Также величина снижения заработных плат зависит от академической сферы: для большинства областей падение составляет 0.5\%, в то время как бухгалтерский учёт и экономика демонстрируют более 1\% ежегодно. 
	
	Далее Michael R. Ransom оценивает собственную модель на данных университета Аризоны и приходит к выводу: заработные платы падают примерно на 1\% c ежегодным ростом сеньорити - эффект более значимый в сравнении с упомянутыми ранее исследованиями. 
	
	После всех описанных выше эмпирических результатов Michael R. Ransom заключает, что действительно можно говорить о существовании отрицательной связи заработной платы и сеньорити. Автор приводит различные объяснения замеченного феномена. Одно из которых - более способные профессора имеют множество альтернатив, в то время как менее способные вынуждены оставаться на прежнем рабочем месте. Эта идея и лежит в основе моделей Milton Harris and Bengt Holmstrom (1982) и Lazear (1986). Поскольку первоначально фирма не знает о способностях работника, она предлагает стартовую заработную плату. В результате некоторые работники остаются недооценёнными. Другие фирмы идентифицируют таких работников и стараются их нанять, предлагая более высокую заработную плату. Получается, что более «высококачественные» работники получают более высокую заработную плату, а работники с высоким сеньорити имеют тенденцию быть «низкокачественными» и получать более низкую заработную плату. Тем не менее остаётся неясным, каким образом измерять «качество» работников. В качестве одного из способов оценки продуктивности авторы предложили число публикаций. Оказалось, что профессора с высоким сеньорити несильно отличаются количеством публикаций о тех, кто работает недолго в данном университете. Авторы пришли к выводу, что продуктивность или же «качество» работников не может служить адекватным объяснением отрицательной зависимости заработных плат и продуктивности.
	
	Продолжая поиски ответа на поставленный вопрос, Michael R. Ransom строит модель монопсонистической дискриминации. Он предполагает наличие одного университета в городе, поэтому для смены работы профессору необходимо переехать, что влечёт за собой определённые издержки. Если издержки высоки, агент предпочтёт остаться в прежнем городе (продолжительность работы в одном университете будет расти), а заработная плата будет ниже возможной предложенной университетом в каком-либо другом городе. Однако данная модель не предсказывает, что зарплаты будут падать с ростом сеньорити. Оценённый коэффициент регрессии смещён, поскольку мы не можем включить в неё ненаблюдаемую компоненту издержек переезда. 
	
	Теперь, когда мы пронаблюдали на монопсоническом рынке эффект падения заработных плат с увеличением сеньорити на примере различных эмпирических исследований и рассмотрели математическую модель, встаёт вопрос: будет ли иметь место рассматриваемый феномен на рынке олигопсонии? Сперва обратимся к исследованию, которое в принципе отрицает существование выявленного феномена, затем уделим внимание особенностям олигопсонического рынка.
	
	Позднее вышла статья авторов William J. Moore, Robert J. Newman, Geoffrey K. Turnbull (1998)\footnote{Robert J. Newman William J. Moore and Geoffrey K. Turnbull. Do academic salaries decline with seniority? Journal of Labor Economics, 16(2):352–366, 1998.}, получивших совершенно иные результаты. Отрицательная зависимость  между заработной платой и сеньорити полностью исчезла, когда в модель включили более исчерпывающие оценки продуктивности индивидов, такие как количество публикаций, цитируемость и многие другие. В качестве выборки авторы использовали профессоров, нанятых на программы Ph.D по экономике в 9 государственных университетов. Для усовершенствования предыдущих моделей продуктивность профессоров стали измерять в трёх сферах: исследования, преподавание и управление. Каждой из них был присвоен свой вес в зависимости от целевой функции кафедры. При этом вводилось предположение, что  исследования являлись самой важной компонентой. Для получения градации качества статей, авторы разделили публикации на два типа по отношению к журналу, в котором были напечатаны. Также была введена переменная, отвечающая за цитируемость: как часто другие исследователи ссылались на данную статью. Качество преподавания оценивалось с помощью дамми переменной, которой присваивалась единица в случае, если профессор когда-либо получал награду за преподавание. И последняя оценка продуктивности - количество лет на должности заведующего кафедрой в высшем учебном заведении как показатель управленческих навыков. Также в модель были включены дополнительные дамми переменные, такие как пол индивида, язык страны (английский-1/не английский-0) и рейтинг университета, где получал Ph.D (топовый-1/ не топовый-0). 
	
	В результате, базовая регрессия оценённая на опыт и синьорити даже с добавлением количества публикаций выявляет отрицательную зависимость заработных плат и продолжительности времени работы в конкретном университете. Однако по мере расширения модели и добавления в неё других вышеописанных переменных наблюдаемая ранее связь исчезает. Профессора с большим сеньорити получают заработную плату относительно меньшую лишь потому, что многие из них менее продуктивны в сравнении в теми, у кого сеньорити ниже.
	
	Теперь обратимся к статье \cite{2}\footnote{Alan Manning V. Bhaskar and Ted To. Oligopsony and monopsonistic competition in labor markets. Journal of Economic Perspectives, 16(2):155–174, 2002.}, которая поможет нам понять природу олигопсонии. Авторы говорят,  что монопсония - крайне нереалистичное предположение о функционировании рынка труда; олигопсония же представляет собой более подходящую структуру. Она описывает ситуацию, когда работодатели обладают рыночной силой, несмотря на то, что конкурируют друг с другом. 

	Одним из источников появления рыночной силы может служить разнородность заработных плат для работников одинаковой квалификации. Например, в одной отрасли работники могут получать больше чем в другой, обладая одними и теми же навыками (что может быть связано с величиной предельного продукта в отраслях). Этот вопрос исследовался в статьях Krueger and Summers (1988) и Gibbons and Katz (1992).
Другим источником рыночной силы может служить существование издержек агента по смене работы. Более того, работники с одинаковыми навыками и способностями могут иметь различные предпочтения, касаемо таких неоплачиваемых характеристик работы как: специфика работы, часы работы, расстояние от дома до работы, атмосфера в коллективе. Поэтому небольшое уменьшение заработной платы необязательно влечёт за собой увольнение.
Авторы статьи объясняют, каким образом модель олигопсонии на рынке труда помогает углубить понимание различных феноменов: плата работодателями за тренинги для работников, дискриминация заработных плат по расам, влияние минимальной заработной платы на занятость, различие в заработных платах для одинаково квалифицированных работников. Подводя итог, V. Bhaskar, Alan Manning and Ted To пишут, что рыночная сила появляется за счёт различий в предпочтениях работников, издержек переезда и неполной информации. Поэтому кажется полноправным заключить, что модель олигопсонии хорошо описывает рынок труда, в частности академический, исследованием которого мы и займёмся.

\section{Модель}

Многие статьи рассматривают академический рынок труда как монопсонию. Мы утверждаем, что существуют также и рынки олигопсонического типа, и совсем не очевидно, что на них будет наблюдаться тенденция к понижению заработной платы. Также согласно статье \cite{2} олигпсония может рождаться в результате появления издрержек на переезд, что напрямую относится к рынку профессоров. В связи с этим мы считаем, что предпосылка о таком виде рынка имеет место быть.  Мы рассмотрим модель олигпсонического рынка труда, в рамках которой у профессора будет выбор: либо остаться в своем регионе и пойти в один из университетов, либо уехать и понести издержки переезда. Модель будет строится на основе модели монопсонического рынка из статьи Seniority and Monopsony in the Academic Labor Market \cite{1}. Из неё мы возьмём базовую концепцию равновесия и добавим новых игроков. 

Пусть существует рынок в рамках одного города, где присутствует $n$ университетов, предъявляющих спрос на одних и тех же академических работников. Это важное замечание, так как несколько ВУЗов могут находится территориально близко, но быть абсолютно разного уровня, в следствие чего они не будут предъявляться спрос на одних и тех же учёных. Обычно в США в одном городе находится лишь несколько ВУЗов, которые могли бы предъявлять спрос на одну и ту же рабочую силу, но для общности мы пока буем предполагать, что их $n$. Рынок академических работников основан в своем большинстве на личном взаимодействии, так как преподаватели являются единственными экземплярами, предлагающими именно такую комбинацию способностей. Это сильно отличает академический рынок труда от многих других (неплохо бы ссылку). Вследствие этого каждый университет выбирает, какую зарплату он предложит отдельному работнику. Зарплату, предложенную $i$ университетом, мы будет обозначать $w_i$. Также будет существовать рыночная заралата $w_m$; именно с такой зарплатой университету придется нанять работника, если профессор, которому предложили зарплату $w_i$, откажется. Университет минимизирует свои ожидаемые издержки: 

\[
EC(w_i) = p_i(w_m, w_i, m, w_{j \neq i})w_i + (1 - p_i(w_m, w_i, m, w_{j \neq i}))w_m
\]

Как мы видим, в зависимости от предложенной зарплаты, зарплаты предложенной другими университетами, зарплаты по рынку и издержкам переезда будет устанавливаться некоторое распределение, случайным исходом для которой будет являться выбор, сделанный преподавателем. Как и в модели с монопсонией мы будем считать для простоты, что агенты обладают полной информацией, и единственный источник случайности заключается в выборе профессором одного из университетов. Мы можем предположить, что функция $p_i ( w_m, w_i, m, w_{j \neq i}) $ имеет следующий вид: 
\[
p_i( w_m, w_i, m, w_{j \neq i}) = \frac{p_i(w_i - w_m + \delta m) w_i}{\#\{w_j : w_i = w_j\}}\mathds{1} \{ w_i \geq w_ {i \neq j}\}
\]
Профессор сравнивает предложенную ему зарплату некоторым университетом с зарплатой рынка и своими дисконтированными издержками переезда. Более того, ввиду наличия функции индикатора максимальной цены преподаватель будет рассматривать лишь тот университет, который предложил ему наибольшую зарплату. Всем остальным ВУЗам придется нанимать преподавателя на общем рынке. По причине присутствия ненаблюдаемых факторов решение всё равно остается случайным. 


Возможен также случай, когда несколько университетов предложат максимальную зарплату. Именно этот фактор учитывает знаменатель дроби, который означает, что вероятность попасть в конкретный ВУЗ делится на количество ВУЗов, предложивших такую же зарплату. 

Проанализируем поведение стороны спроса на рынке. В статье Olygopsony and Monopsonistic Competition in Labor Markets Bhaskar, A. Manning, Ted To (2002)  рассматриваются различные варианты взаимодействия игроков на олигпсоническом рынке.  Мы будем предполагать, что они не вступают в сговор. Мы считаем, что такая предпосылка вполне оправдана, так как профессора являются единичным товаром, вследствие чего доход от его приобретения почти невозможно разделить легальным путем. 

Заметим, что если ВУЗ устанавливает зарплату ниже чем его конкуренты, то он сталкивается с рыночной ценой с вероятностью 1, соответсвенно в симметричном равновесии  вузы установят одинаковую зарплату. Также должно быть выполнено следующее условие для каждого конкретного учебного заведения:
\[
p_i(w_i - w_m + \delta m) w_i + (1 - p_i(w_i - w_m + \delta m))w_m = \frac{p_i(w_i - w_m + \delta m) w_i}{n} + (1 -\frac{ p_i(w_i - w_m + \delta m)}{n})w_m \leq w_m
\]
Это уравнение означает, что конкретному ВУЗу невыгодно увеличивать зарплату на любую положительную величину для повышения вероятности прихода к нему профессора. Более того, в равновесии ожидаемые издержки отдельного ВУЗа не должны превышать рыночную цену на учёного, иначе он обязательно переключится на эту альтернативу.
Раскрывая скобки, мы получаем: 

\[
\frac{(n-1)p_i(w_i - w_m + \delta m) w_i}{n} = \frac{(n-1) p_i(w_i - w_m + \delta m) w_m}{n}
\]
\[
w_i = w_m
\]

Мы видим, что в случае такой конкуренции будет установлено равновесие, в котором каждый ВУЗ предлагает одинаковую зарплату, равную рыночной. Ученый в свою очередь с вероятностью $p(\delta m)/n$ выберет один из ВУЗов. Тогда вероятность, что он не выберет ни один из вузов и будет равна:
\[
1 - \sum_{i = 1}^n p_i (\delta m)/n = 1 - p_i (\delta m)
\]
Заметим также, что теперь ВУЗам безразлично, какого профессора нанимать: с конкурентного рынка или с внутреннего, поэтому равновесие будет включать в себя даже такие ситуации, когда все ВУЗы кроме одного нашли профессора на общем рынке. Тогда расширяется ряд случаев, которые мы можем анализировать в действительности. Естественно также предположить, что не за каждого профессора идёт такая борьба на рынке труда, следовательно, на данных мы ожидаем увидеть по-прежнему отрицательную взаимосвязь как и в предыдущих статьях, но более слабую.

Аналогично монопсонической модели мы можем предположить, что в течение жизни профессор через равные промежутки времени совершает выбор: уехать или остаться. Его зарплата в этот момент обновляется в результате реализации случайного события. Именно на основании такого анализа авторы статьи \cite{1} и пришли к выводу о падении зарплат. Также мы можем предположить, что конкуренция за профессора не ведется на протяжении всех периодов, что по нашему мнению является вполне реалистичной предпосылкой, так как профессора редко меняют университет. Исходя из этого можно заключить, что и в конкуренцию за профессоров ВУЗы вступают редко. Следовательно, в какие-то периоды мы будем снова оказываться в ситуации монопсонии. Тогда на протяжении всех периодов зарплата будет вести себя скачкообразно, если профессор всё это время оставался в городе. Если же он уехал из города, то его персона не представляет интереса для нашего анализа. 

Мы можем показать данный факт более формально введя вероятность конкеренции за профессора и положив её равной $g$, тогда за $t$ периодов ожидаемое количество скачков зарплаты будет равно $gt$, и соответсвенно столько же раз в среднем будет подскакивать зарплата профессора, в остальные же периоды она будет меньше рыночной в соответсвии с выводом авторов статьи \cite{1}. Мы ещё раз отмечаем, данные расчеты верны лишь при условии, что профессор не уезжает из города. 

Рассматривая данный рыночный механизм в динамике мы пришли к выводу, что эффект понижения зарплат на олигопсонических рынках будет наблюдаться, являясь более слабовыраженным, чем на рынках, где присутствует лишь один университет. Это будет связано с тем, что конкуренция за профессоров, по нашему мнению, явление не непрерывное: она будет оживляться в одни периоды, а в другие - угасать.

\section{Кейс}

В качестве примера, иллюстрирующего нашу модель, будет рассмотрена Хьюстонская система университетов. Во-первых, университеты внутри этой системы независимы и примерно одного ранга (имеют оценку tier 2), то есть можно предположить, что между ними существует конкуренция за рабочую силу. Во-вторых, в пределах города Хьюстон присутствует большое количество университетов второго ранга, например, Техас Саутерн и Университет Конкордия. Так как в Хьюстоне присутствуют университеты одного ранга и перемещение между ними не сопряжено с издержками, можно сделать вывод, что рынок академического труда города Хьюстон можно рассматривать как олигопсонию. Заметим, что это справедливо только для университетов второго ранга и ниже, потому что в данном городе присутствует только один университет первого ранга — Райс.

Данные о работниках Хьюстонской системы находятся в открытом доступе на сайте: \url{http://salaries.texastribune.org/university-of-houston/}. Перед началом анализа данные были обработаны. Сначала была проведена очистка данных от работников департаментов, занимающихся административной и обслуживающей деятельностями. Далее были оставлены только работники, занимающиеся преподаванием, то есть профессоры. Наконец, была добавлена переменная $seniority$, показывающая сколько полных лет индивид проработал в данном университете.
Для анализа будут использованы следующие переменные из ранее упомянутой базы данных: $rate$ --- уровень заработной платы, $sex$ --- пол индивида (1 --- мужчина), $Fullp$ --- является ли индивид профессором, работающим на постоянной основе, $Type$ --- полный рабочий день или частичный, $Foreign$ --- показывает расу (0 --- европеоидная, 1 --- остальные). Обработанные данные доступны по ссылке: \url{https://goo.gl/B8MTpk}

Дескриптивная статистика выборки:

\begin{center}
\begin{tabular}{|c|c|c|}
\hline
• & lograte & seniority \\
\hline
Среднее & 11.47 & 13.35 \\
\hline
Медиана & 11.45 & 10.00 \\
\hline
Стандартное отклонение & 0.39 & 12.11 \\
\hline
N=1016 & • & • \\
\hline
\end{tabular}
\end{center}

\begin{center}
\begin{tabular}{|c|c|c|}
\hline
• & N & \% \\
\hline
Мужчины & 688 & 67.72 \\
\hline
"Полные" профессора & 351 & 34.55 \\
\hline
Работающие полный день & 994 & 97.83 \\
\hline
"Европейцы" & 634 & 62.40 \\
\hline
\end{tabular}
\end{center}

Воспользуемся Минцеровским уравнением для того, чтобы оценить влияние рабочего стажа в одном университете на заработную плату:

$$lograte=\alpha_1+\alpha_2 Sex+\alpha_3 Seniority +\alpha_4 Fullp+\alpha_5 Type +\alpha_6 Foreign$$

где $lograte$ —- логарифм заработной платы.

Значения полученных коэфициентов:

\begin{center}
\begin{tabular}{|c|c|c|}
\hline
• & Оценка коэфициента & Стандартное отклонение оценки \\
\hline
interception & 10.79754 & 0.07216 \\
\hline
sex & 0.12098 & 0.02134 \\
\hline
seniority & -0.00588 & 0.00095 \\
\hline
fullp & 0.51297 & 0.02447 \\
\hline
type & 0.48461 & 0.06923 \\
\hline
foreign & 0.07161 & 0.02056 \\
\hline
\end{tabular}
\end{center}

Все полученные коэфициенты значимы на очень низком уровне значимости: $\alpha<0.001$.

В нашем анализе $R^2=0.36$, в то время как в статье \cite{3} при построении первой модели, которая не включала в себя показатели продуктивности профессоров, $R^2=0.33$. Такое значение позволяет нам полагать, что зависимость однозначно существует, но частично остаток должен объясняться с помощью дополнительных регрессоров, например, количества и качества публикаций, количества лет в администрировании учебных процессов, наличия преподавательских наград и других переменных, описывающих производительность преподавателя. В статье \cite{3} были приведены примеры моделей, в которых при добавлении в регрессию переменных, отвечающих за продуктивность, точность модели увеличивалась до $R^2=0.73$. В нашем кейсе не было использовано подобных переменных из-за отсутствия ресурсов по сбору данных, где количество наблюдений превышало 1000. Вполне вероятно, что в случае добавления в нашу модель переменных, оценивающих продуктивность работников, коэффициент при $seniority$ снизится, как это и произошло в статье \cite{3}. Мы также не имели данных о возрасте рабочих, что введёт к смещению оценок (в уравнении Минцера возраст является одним из важнейших показателей).

Также стоит отметить, что причиной возможных проблем может являться и нерепрезентативность выборки. Такой вывод был сделан исходя из графика плотности логарифма заработной платы: в идеальной ситуации логарифм заработной платы должен быть распределен нормально, однако, здесь наблюдается смещение в сторону более высоких зарплат профессоров.

В итоге, можно сказать, что нам не удалось полностью продемонстрировать результаты модели на данных. Основная причина заключается в том, что мы не имели достаточно данных для более полной спецификации модели.



\begin{center}
\includegraphics[scale=0.5]{image1}
\end{center}

\section{Заключение}

Подводя итог, мы можем сказать, что нам частично удалось достичь поставленных целей. Мы построили и проанализировали модель олигопсонического рынка труда в результате чего пришли к выводу о более слабой отрицательной зависимости заработной платы от стажа работы среди профессоров, чем на монопсонических рынках труда.  По нашему мнению мы можем рассматривать статью \cite{3}, как косвенное подтверждение нашей модели, результаты которой согласуются с тем, что получили мы: стаж практически не будет влиять на заработную плату.   Мы также считаем, что полученные результаты имеют смысл, так как нам кажется не правильным предполагать, что вся рыночная власть находится в руках университета. Профессор --- это эксклюзивный товар, в результате чего университеты могут вступать в конкуренцию, образуя тем самым олигопсонический рынок.  Анализ данных не позволяет нам сделать однозначных выводов, ввиду недостающей информации. Поэтому в данном направлении необоходимо провести дополнительное исследование, чтобы проверить  гипотезу о том, что структура рынка может влиять на наблюдаемый эффект. 

Многие предыдущие результаты косвенно согласуются с тем, что получили мы. Например в \cite{1} говориться о том, что в сильных ВУЗах эффект падения зарплат более выражен. С нашей этот факт можно объяснить тем, что как правило на один город присутствует не более одного сильного ВУЗа. Следовательно образуется монопсонический рынок труда для профессоров. 


Если удасться подтвердить гипотезу о связи олигопсонии и более слабой зависимости, то можно провести отдельное мета --- исследование на основании предыдущих статей, потому что различные результаты полученные исследователями в прошлом также могут быть объяснены тем, что одни использовали данные олигпсонических рынков, а другие монопсонических. 

\renewcommand{\refname}{Список литературы}
\bibliographystyle{named}
\bibliography{biblio} 

\end{document}



