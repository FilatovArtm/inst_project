\documentclass[a4paper, 12pt]{article}
\usepackage{fontspec} 
\usepackage[utf8]{inputenc}
\usepackage[russian,english]{babel}
\setmainfont{Times New Roman}


 % для компиляции теперь нужно использовать XeLaTex 
\usepackage{amsfonts}
\usepackage[paper=a4paper,top=13.5mm, bottom=13.5mm,left=16.5mm,right=13.5mm,includefoot]{geometry}
\usepackage{indentfirst}
\usepackage{dsfont}
\usepackage{graphicx} 
\graphicspath{{images/}} 
\setlength\fboxsep{3pt} 
\setlength\fboxrule{1pt} 
\usepackage{wrapfig} 
\usepackage{amsmath}
\usepackage{amsthm}
\renewcommand{\proofname}{Доказательство}

\DeclareMathOperator*{\argmin}{arg\,min}
\newcommand{\inde}{\perp\!\!\!\perp}


\theoremstyle{definition}
\newtheorem{definition}{Определение}[section]


\theoremstyle{plain}
\newtheorem{lemma}{Лемма}[section]
\newtheorem{svoistvo}{Свойство}[section]
\newtheorem{theorem}{Теорема}[section]


\title{Исследовательский проект по инстуциональной экономике}
\date{\today}

\begin{document}
\maketitle

\noindent{\bf Аннотация }-- ...

\section{Введение}
Мы устверждаем, что в существуют рынки на которых присутствует сразу несколько игроков со стороны спроса на труд в одном регионе. 
\section{Обзор литературы}

\section{Модель}

Многие статьи рассматривают академический рынок труда в рамках одной местности (изменить этот слово), как монопсонию (ссылки на статьи). Мы рассмотрим модель олигпсонического рынка труда в рамках которой у профессора будет выбор либо остаться в своем регионе и пойти в один из университетов, либо уехать, но понести издержки переезда. Модель будет основана на статье ( бла бла). 

Пусть существует рынок в рамках одного города, где присутствует $n$ университетов одного уровня, предъявляющие спрос на одних и тех же академических работников (смищно). Каждый университет может предлагать разную зарплату разным профессорам. Зарплату предложенную $i$ университетом мы будет обозначать $w_i$. В случае, если профессор отказывается, то университет несет издержки равные зарплате профессора с конкурентного рынка равные $w_m$ (переписать этот кусок, потому что копи паст) (Возможно вставить сначала предпосылки относительно поведения профессора). Университет минимизирует свои ожидаемые изержки: 
\[
EC(w_i) = p_i(w_m, w_i, m, w_{j \neq i})w_i + (1 - p_i(w_m, w_i, m, w_{j \neq i}))w_m
\]
Как и в модели ( ... ) мы будем предлагать для простоты, что агенты обладают полной информацией, и единственный источник случайности заключается в выборе профессора одного из университетов. Мы можем предложить, что функция $p_i (\ldots) $ имеет следующий вид: 
\[
p_i( w_m, w_i, m, w_{j \neq i}) = p_i(w_i - w_m + \delta m) \mathds{1} \{ w_i \geq w_ {i \neq j}\}
\]
Смысл индикатора заключается в том, что единственный вариант, который рассмотрит профессор, это тот, который предложит ему наибольшую зарплату. В остальном функция повторяет таковую в статье ( ).  И основной смысл в её виде заключается в том, что профессор сравнивает наибольшую зарплату, которую ему предложили в данном городе, с зарплатой на рынке и принимает решение осонованное на ненаблюдаемых нами факторах, а следовательно на статистическом уровне мы можем интерпретировать его как случайное. 

В случае если максимальных зарплат несколько, то вероятность того, что профессор придет в университет в своем городе нормируется на количство вузов предложивших наибольшую зарплату. 

Проанализируем поведение олигополистов на рынке. Мы будем предполагать, что они не вступают в сговор. Если бы они вступили в сговор, то для каждого конкретного профессора ситуация привратилась бы монопсонию и вывод бы не отличался от таковых в других  статьях. 

Заметим, что если вуз устанавливает зарплату ниже чем его конкуренты, то он сталкивается с рыночной ценой с вероятностью 1, соответсвенно в симметричном равновесии  вузы должны установить одинаковую зарплату. И должно быть выполнено следующее условие для каждого конкретного учебного заведения
\[
p_i(w_i - w_m + \delta m) w_i + (1 - p_i(w_i - w_m + \delta m))w_m = \frac{p_i(w_i - w_m + \delta m) w_i}{n} + (1 -\frac{ p_i(w_i - w_m + \delta m)}{n})w_m
\]
То есть каждому университету не выгодно не сколько либо увеличить предлагаемую зарплату, чтобы повысить вероятность прихода к нему данного работника. Раскрывая скобки мы получаем 
\[
\frac{(n-1)p_i(w_i - w_m + \delta m) w_i}{n} = \frac{(n-1) p_i(w_i - w_m + \delta m) w_m}{n}
\]
\[
w_i = w_m
\]



\section{Кейс}

\section{Заключение}




\end{document}



